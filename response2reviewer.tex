\documentclass{ar2rc}
\usepackage{multicol}
\usepackage{tcolorbox}
\usepackage{booktabs}
\usepackage{array}
\usepackage{geometry}
\usepackage{float}  % 强制后续文本不超过前面的图片,\begin{figure}[H]

\usepackage[ruled]{algorithm2e}  % algorithm

\usepackage{natbib}

% 关联主文档用来索引引用
\usepackage{xr}
\externaldocument{main} 

\usepackage[draft,commandnameprefix=ifneeded]{changes}
\definechangesauthor[color=red]{R1}
\definechangesauthor[color=blue]{R2}

\rfoot{Page \thepage}

\lettertitle{The Response to Reviewers}
\title{GrapeCPNet: A Self-supervised Point Cloud Completion Network for 3D Phenotyping of Grape Bunches}
\author{
    Wenli Zhang \textsuperscript{a,*},
    Chao Zheng \textsuperscript{a},
    Chenhuizi Wang \textsuperscript{a},
    Pieter M. Blok \textsuperscript{b},
    Haozhou Wang \textsuperscript{b},
    Wei Guo \textsuperscript{b,*}
}
\journal{Computers and Electronics in Agriculture}
\doi{COMPAG-D-25-00311R1}

\corresponding{Corresponding author: zhangwenli@bjut.edu.cn; guowei@g.ecc.u-tokyo.ac.jp }
\email{
    \textsuperscript{a} Information Department, Beijing University of Technology, Beijing, China \\
    \textsuperscript{b} Graduate School of Agricultural and Life Sciences, The University of Tokyo, Tokyo, Japan
}

\begin{document}

\begin{center}
    \maketitle
\end{center}

%%%%%%%%%%%%%%%%
% cover letter %
%%%%%%%%%%%%%%%%
\thedate

Dear Editor:

Thank you for giving us the opportunity to submit a revised draft of the manuscript ``\thetitle'' for publication in the Journal of \thejournal. We appreciate the time and effort that editors and the reviewers dedicated to providing feedback on our manuscript and are grateful for the insightful comments and valuable improvements to our paper. We have incorporated most of the suggestions made by the reviewers. Those changes are highlighted within the manuscript. Please see below, for a point-by-point response to the reviewers' comments and concerns. All page numbers refer to the revised manuscript file with tracked changes.

Thank your for your consideration. I am looking forward to hearing from you soon.

Sincerely,

Wei Guo\\
Associate professor\\
Graduate School of Agricultural and Life Science\\
The University of Tokyo, Tokyo, Japan\\
Email: guowei@g.ecc.u-tokyo.ac.jp

\vfill
\textbf{Note:} To enhance the legibility of this response letter, all the editor's and reviewers' comments are typeset in boxes. Rephrased or added sentences are typeset in color. The respective parts in the manuscript are highlighted to indicate changes.

%====================
% Response to Editor
%====================
\editor

%%%%%%%%%%%%%%%%%%%%%%%%%%%%%%%%%%%%%%%%%%%%%%%%%%%%%%%%%%%%%%%
\begin{reviewercomment}
    Thank you for submitting your manuscript to Computers and Electronics in Agriculture. 
    I have received comments from reviewers on your manuscript. 
    Your paper should become acceptable for publication pending suitable moderate revision and modification of the article in light of the appended reviewer comments.

    When resubmitting your manuscript, please carefully consider all issues mentioned in the reviewers' comments, outline every change made point by point, and provide suitable rebuttals for any comments not addressed.
\end{reviewercomment}

\response{
    We sincerely appreciate the positive feedback from the editor and reviewers regarding our manuscript. 
    We are grateful for the opportunity to revise our work based on the valuable comments received. 

    In this revision, we have:

    1. Carefully addressed all reviewer comments point-by-point;\\
    2. ...
    % 2. Provided detailed responses explaining each modification;\\
    % 3. Made all necessary revisions to improve manuscript clarity and technical accuracy;\\
    % 4. Included tracked changes in the revised manuscript for transparent review.

    We believe these revisions have significantly strengthened the paper, and we appreciate the editor's and reviewers' time and constructive suggestions. 
    Please find our point-by-point responses to each reviewer comment in the following sections.
}

%============
% Reviewer 1
%============
\reviewer
%%%%%%%%%%%%%%%%%%%%%%%%%%%%%%%%%%%%%%%%%%%%%%%%%%%%%%%%%%%%%%%
\begin{reviewercomment}
    Overall, the paper is well structured with a clear goal, and the authors provide insight to justify their design choices. 
    However, I believe the paper will benefit from  more experiments, particularly on the instance segmentation and completion parts.
\end{reviewercomment}

\response{
    We sincerely appreciate the reviewer's positive assessment of our study and manuscript. 
    Regarding the suggestion for additional experiments, we agree that further validation would strengthen the work. 

    In response, we have:

    ...
}

\todoblock{
    Summary of added experiment.
}

%%%%%%%%%%%%%%%%%%%%%%%%%%%%%%%%%%%%%%%%%%%%%%%%%%%%%%%%%%%%%%%
\begin{reviewercomment}
    Regarding the instance segmentation, I understand that this is not the main contribution of the paper. Still, it would be nice to test different approaches here to evaluate the robustness of the completion network to varying levels of wrong predictions. Along a similar line, I would have liked to see the completion network working on ground truth segmentation labels as this will disentangle errors coming from the instance segmentation net and the completion net.
\end{reviewercomment}

\response{
    We sincerely appreciate the reviewer's valuable suggestions regarding the instance segmentation evaluation. 
    In response, we have conducted additional comparative experiments by applying our completion network to both manually annotated ground truth labels and SoftGroup segmentation results, with quantitative comparisons shown in the new results figure and table below.

    
    A t-test analysis revealed no statistically significant differences between the two approaches (p > xxx), demonstrating that our high-accuracy segmentation network (achieving mAP@[.5:.95] of 0.995 represents an AP50 of 0.999 for our results) provides sufficiently reliable inputs for the completion network. 

    We maintain our design of having specialized networks focus on their respective tasks: the segmentation network achieves reliable segmentation accuracy, while the completion network specializes in partial point cloud completion. 
    Rather than increasing the completion network's robustness to handle poor segmentation results.
    It would require generating artificial training data with various error patterns that might negatively impact training.
    We believe it is more principled to improve the segmentation performance directly. 
    This modular approach ensures clear responsibility separation between network components.
}

\todoblock{
    Execute the following experiment on manual annotation bunches:

    1. Apply completion networks directly on ground truth (separated 3D reconstruction)

    2. Run SoftGroup segmentation on ground truth, and then apply completion networks on these results

    3. Compare two results to draw comparison tables and runing t-test to show no significant differences.
}

\manuscript{
    ...
}

%%%%%%%%%%%%%%%%%%%%%%%%%%%%%%%%%%%%%%%%%%%%%%%%%%%%%%%%%%%%%%%
\begin{reviewercomment}
    Regarding the completion net, I would have liked to see ablation studies here as I believe this is the main  contribution of this work. For example, how does the number of cuts  influence the results of the completion metrics? How does the number of  output points in the centroid-based contour influence the snowflake net?
\end{reviewercomment}

\response{
    We thank the reviewer for these insightful suggestions regarding ablation studies for our completion network. 
    For current manuscript, we already have part ablation studies. 
    The GrapeCPNet is combined the modification of PoinTr and SnowFlakeNet.
    As shown in the original Table 3, we compared the performance of:
    (1) Standalone PoinTr; (2) Standalone SnowFlakeNet; (3) Our modified GrapeCPNet (combined architecture).
    The results demonstrated the superiority of our integrated approach.

    We have also expanded our ablation studies to provide more comprehensive analysis as you suggested:

    1. \textbf{Impact of Cut Numbers}: We conducted new experiments evaluating how different occlusion levels (number of cuts) affect completion performance:

    ...

    The results showed that mixed occlusion training significantly improves generalization

    2. \textbf{Point Number Analysis}: We investigated how output point count affects SnowFlakeNet performance:

    ...

    The result showed higher point counts yield better completion quality.
}

\todoblock{
    To add the experiement:

    1. split training data to 1 cuts, 2cuts, 3cuts, 4 cuts and 5 cuts, train 5 individual networks and validate on val dataset and compare with 1-5 full training data? (expected -> less cuts in training data, worse performance on server overlapped.)
    
    2. Decrease the feature M (current=2048) of networks, 2048, 1024, 512, 256, to see if any affects of completion results.
}

\manuscript{
    ...
}

%%%%%%%%%%%%%%%%%%%%%%%%%%%%%%%%%%%%%%%%%%%%%%%%%%%%%%%%%%%%%%%
\begin{reviewercomment}
    Finally, is there a reason why the approach of Blok et al. 2025 is not being used as a  baseline? I would have liked to see a non-learning-based baseline as  well, like Marangoz et al. 2022
\end{reviewercomment}

\response{
    We appreciate the reviewer's suggestion regarding additional baselines. 

    The work of \citet{marangoz_fruit_2022} generates cube-like point clouds for peppers, which may not be suitable for our sphere-like grape berries.
    Additionally, re-implementation their algorithm is time-consuming and unlikely to benefit our current grape project.
    Our ellipsoid surface fitting is a similar non-learning-based approach, serving for berry traits analysis in this study. 
    Therefore, it may not be appropriate as a baseline comparison.

    The comparison with \citet{blok_highthroughput_2025} was not included initially because these were independent parallel research projects during our manuscript preparation phase. 
    Following editorial recommendations during the submission process, we subsequently added discussions of comparable works in the revised manuscript, including it into the discussion part after its acceptance.

    In this revision, we have now included CoRe++ from \citet{blok_highthroughput_2025} performance on our grape dataset in Table 3. 
    However, we observe that the network architectures target substantially different objectives.
    Our grape dataset involves partial completion from the same sensor type, while the potato network focuses on fusing different sensors 
    (with incomplete point clouds from RGB-D and complete ones from SfM, exhibiting significant shape and texture differences). 
    Our approach emphasizes better completion performance while the \citet{blok_highthroughput_2025} more on processing speed and real-time processing, and consequently, CoRe++ does not demonstrate significant advantages in our grape dataset scenario.
}

\todoblock{
    Add CoRe++ results to Table 3 (collaboration with pieter)
}

\manuscript{
    ...
}
    
%============
% Reviewer 2
%============

\reviewer
%%%%%%%%%%%%%%%%%%%%%%%%%%%%%%%%%%%%%%%%%%%%%%%%%%%%%%%%%%%%%%%
\begin{reviewercomment}
    Line 159: What does "instance-level" mean?
\end{reviewercomment}

\response{
    We appreciate the reviewer's comment. The term "instance-level" refers to the individual segmentation of each grape berry and stem in the point cloud, where each instance is distinctly labeled and separated from others. We have clarified this definition in the revised manuscript.
}

\manuscript{
    In order to train and validate SoftGroup, each berry and stem in the bunch point cloud obtained in Section~\ref{sec:212} was \added[id=R2]{individually} labeled \replaced[id=R2]{(instance-level segmentation) as shown in}{at the instance-level} Fig.~\ref{fig:raw11}.
}

%%%%%%%%%%%%%%%%%%%%%%%%%%%%%%%%%%%%%%%%%%%%%%%%%%%%%%%%%%%%%%%
\begin{reviewercomment}
    Which platform, software, or tool was used for labeling?
\end{reviewercomment}

\response{
    We thank the reviewer for raising this important methodological detail. 
    The 3D point clouds were labeled using CloudCompare (v2.12.4), an open-source software for 3D point cloud and mesh processing.
}

\todoblock{
    Confirm which tool for labeling, or SoftGroup can automatically annotation?
}

\manuscript{
    
}

%%%%%%%%%%%%%%%%%%%%%%%%%%%%%%%%%%%%%%%%%%%%%%%%%%%%%%%%%%%%%%%
\begin{reviewercomment}
    Line 179: What is meant by "selection area"?
\end{reviewercomment}

\response{
    We appreciate the reviewer's question. 
    The term "selection area" refers to the regions where portions are removed from the complete berry point cloud to simulate occlusion. 
    To improve clarity, we have revised the terminology to "removal sphere regions" throughout the manuscript, as this more accurately describes the spherical regions where point cloud removal occurs during the incomplete data generation process.
}

\manuscript{
    The inputs include a single complete berry point cloud and the maximum number of \replaced[id=R2]{removal sphere regions}{ selection areas (5 in this paper)}. 
}

\manuscript{
    Supervised point cloud completion training requires the incomplete-complete data pairs of the same object to build a mapping relationship.
    The common incomplete characteristics of the berry are shown in Figure~\ref{fig:raw4}.
    In this paper, \added[id=R2]{to decrease the labor cost of such data pair generation,} the complete grapes were obtained by single berry reconstruction (Section~\ref{sec:212}) and were used as ground truth.
    While the incomplete grapes were \replaced[id=R2]{generated by removing parts overlapped with randomly generated 3D sphere regions}{cut} from the complete grapes \deleted[id=R2]{following characteristic of occlusion and were used as training inputs.} 
    \deleted[id=R2]{This data pair generation strategy alleviated the data annotation efforts.}
    \deleted[id=R2]{Based on the incomplete characteristics, we developed a selection method to generate the incomplete berry point cloud from the complete berry point cloud.}
}

\manuscript{
    The pseudo-code of our proposed \added[id=R2]{removal} algorithm is shown in Algorithm \ref{alg:1}. 
    The inputs include a single complete berry point cloud and the maximum number of \replaced[id=R2]{removal sphere regions}{selection areas (5 in this paper)}. 
    The output is a selected incomplete berry point cloud. 
    First, the single complete berry point cloud was normalized and centralized (Fig.~\ref{fig:raw12}a). 
    Second, randomly \replaced[id=R2]{choose the number of removal sphere regions for current grape berry}{select the number of cuts}, with a maximum of five \replaced[id=R2]{removal sphere regions}{cuts as} specified in this paper. 
    Then, for each \replaced[id=R2]{removel sphere region,}{loop of selection, a 3D sphere model was generated in the space to collide with the berry point cloud, and} the collision portion \added[id=R2]{with the berry point cloud} was removed \deleted[id=R2]{to complete the operation}. 
    The \deleted[id=R2]{maximum} center and radius parameters of the 3D \replaced[id=R2]{removal sphere region}{sphere model} were \replaced[id=R2]{randomly}{manually} set according to the \added[id=R2]{normalized} berry size. 
    In this paper, we set \replaced[id=R2]{the sphere center positions within a random range of $\pm 0.75$ relative to the grape center, and the sphere radii within a random range of 0 to 0.25.}{a maximum of 0.75 for the center offsets and 0.25 for the radius offsets.} 
    \replaced[id=R2]{The incomplete berry point cloud was generated as training data by iteratively removing spherical regions.}{Continue the previous selection looping until the limits are met. The selected incomplete point cloud was saved from each loop as training data} (Fig.~\ref{fig:raw12}b). 
}

\begin{algorithm}
    \caption{The selection method for generating training data of incomplete berries}
    \label{alg:1}
    \KwData{$P_C$ \tcp{one complete berry point cloud}}
    \KwData{$n$ \tcp{\added[id=R2]{number of incomplete berries to generate}}}
    \KwResult{$P_{\text{output}} = \{P_{o_i} \mid i = 1, \cdots, n\}$ \tcp{incomplete berry point cloud sets} }
    
    $\hat{P}_C \gets \text{Normalize}(\text{Translate}(P_C))$ \tcp{center to $(0,0,0)$, range to $[-0.5m, 0.5m]$}
    \For{$i = 0 \to n$}{
        set $m = \text{random}\{1,2,3,4,5\}$ \tcp{\added[id=R2]{number of removal sphere regions}}
        $P_{o_i} \gets copy(\hat{P}_C)$ \tcp{initialize one output}
        \For{$j = 0 \to m$}{
            set $O = \{(x_o, y_o, z_o) \mid x, y, z \in \text{random}[-0.75, 0.75]\}$ \tcp{\added[id=R2]{removal sphere} center} 
            set $R = \text{random}[0.25, 0.75]$ \tcp{\added[id=R2]{removal sphere} radius}
            set $M_S \gets \text{sphere}(O, R)$ \tcp{generate \added[id=R2]{removal sphere region}}
            $P_{o_i} = P_{o_i} - P_{o_i} \cap M_S$ \tcp{remove overlap \added[id=R2]{within sphere region}}
        }
        $P_{\text{output}_i} \gets P_{o_i}$ \tcp{add to output set}
    }
\end{algorithm}

\begin{figure}[H]
    \centering
    \includegraphics[width=1\textwidth]{figures/Figure9.pdf}
    \caption{\replaced[id=R2]{Workflow of automatic paired training dataset generation}{Example of dataset composition} for the \deleted[id=R2]{self-supervised training-based method of the} berry point cloud completion network.}
\end{figure}

%%%%%%%%%%%%%%%%%%%%%%%%%%%%%%%%%%%%%%%%%%%%%%%%%%%%%%%%%%%%%%%
\begin{reviewercomment}
    Table 3 results: Was the same dataset used to assess the other algorithms, or were those metrics taken from their original studies?
\end{reviewercomment}

\response{
    We appreciate the reviewer's attention to experimental rigor. 
    To ensure fair comparison, all the comparing methods were conducted on our grape dataset under the same conditions.
    We have added this clarification to the results part.
}

\manuscript{
    GrapeCPNet achieves the best results under the self-supervised training method in both the training phase and the application phase, better than the comparative algorithms \added[id=R2]{under the same condition of our grape dataset}. 
}


%%%%%%%%%%%%%%%%%%%%%%%%%%%%%%%%%%%%%%%%%%%%%%%%%%%%%%%%%%%%%%%
\begin{reviewercomment}
    Occlusion threshold: Is there a berry-occlusion threshold? 
    For example, if less than 50\% of a berry is visible, is it included? How are barely visible berries handled?
\end{reviewercomment}

\response{
    We thank the reviewer for this insightful question regarding occlusion handling. 
    Our approach was designed specifically to address challenging occlusion scenarios.

    We intentionally avoided setting visibility thresholds (e.g., 50\% cutoff) to maintain the applicable of proposed method.
    As shown in Fig.~\ref{fig:raw12}b, after removing 4 sphere regions, the incomplete berry for network training can achieved nearly around 75\% occlusion. 
    Thus, the GrapeCPNet is able to learn serever occlusions for barely visible berries.

    However, we must also admit that when severely obscured, only partial information can be seen. 
    The more obscured it is, the less information it provides. 
    Therefore, errors inevitably increase when the berry is not in a regular shape.
    As stated in the discussion part, we planed to taking prior knowledge of berry geometry to further improve the quality of completion.
}

\manuscript{
    We found that GrapeCPNet performed a little bit worse in the case of strange berry shapes, such as drop-shaped \added[id=R2]{berries, particularly when berries were sereverly occluded}.
    We subsequently consider utilizing the prior knowledge of berry geometry to achieve more detailed morphological feature extraction and representation to improve the quality of completion. 
}


%%%%%%%%%%%%%%%%%%%%%%%%%%%%%%%%%%%%%%%%%%%%%%%%%%%%%%%%%%%%%%%
\begin{reviewercomment}
    Limitations \& future directions: How will the  model perform on wine grapes. Wine grapes accounts for ~60\% of US grape  production; clusters are denser, and berries are smaller than in table grapes.
\end{reviewercomment}

\response{
    We appreciate the reviewer's insightful comment regarding wine grapes, which indeed represent a major segment of grape production. 
    In our current study, we just focused on table grape varieties commonly found in Chinese markets.
    Although they covered a wide range of cluster densities (from sparse to very dense) and berry sizes (20-35 cm axis length, as shown in Table~\ref{tbl:1} and Figure~\ref{fig:raw17}), 
    we acknowledge that wine grapes typically have smaller berry sizes lower than 13-15 mm \citep{manso_wine_2021,melo_berry_2015}, these small sizes were not included in our dataset.

    From a deep learning perspective, our model architecture should be capable of handling smaller wine grapes if provided with appropriate training data, as the fundamental detection principles remain similar regardless of grape variety. 
    Nevertheless, we agree that future work should specifically address wine grape detection through collaboration with U.S. researchers to: 
    (1) collect wine grape datasets, and 
    (2) evaluate whether models trained on table grapes can generalize well to the smaller wine grape berries without additional training.

    We have added the previous points to the discussion parts.
}

\manuscript{
    \added[id=R2]{
        This study has another limitation: it focuses only on four table grapes, which are currently common and available in Chinese markets.
        Although we examined a wide range of cluster densities (from sparse to very dense) and berry sizes (20-35 mm axis length, as shown in Table~\ref{tbl:1} and Figure~\ref{fig:raw17}),
        the study did not include wine grapes, which account for over 60\% of U.S. grape production and typically have smaller berries and denser clusters.
        As \citet{manso_wine_2021,melo_berry_2015} report, wine grape berries are often smaller than 13-15 mm, a size range not covered in our dataset.
        From a deep learning perspective, our model architecture should be able to detect smaller wine grapes with appropriate training data, as the basic detection principles are similar across grape varieties.
        Future work should involve collaboration with U.S. researchers to collect wine grape datasets and assess whether models trained on table grapes can generalize well to smaller wine grape berries without additional training.
    }

    \replaced[id=R2]{Additionally, this study only}{Another limitation of this study is that it} focused on a single berry bunch in a controlled environment.
}



% \phantom{\cite{}} % 添加一个不可见的伪引用, 不会显示但能满足BibTeX要求避免无引用时报错
\bibliographystyle{elsarticle-harv}
% \renewcommand{\bibsection}{} % 隐藏reference标题,如果需要则注释掉这一行
\bibliography{references}

\end{document}

% The copy-paste templates
\begin{verbatim}

    %%%%%%%%%%%%%%%%%%%%%%%%%%%%%%%%%%%%%%%%%%%%%%%%%%%%%%%%%%%%%%%
    \begin{reviewercomment}
        ...
    \end{reviewercomment}
    
    \response{
        ...
    }
    
    \manuscript{
        ...
    }
    
\end{verbatim}