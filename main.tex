\documentclass[12pt]{article}
\usepackage{setspace} 
\usepackage{geometry}
\usepackage{enumitem}
\usepackage{lipsum}
\usepackage{hyperref}
\usepackage{lineno}

%\usepackage[utf8]{inputenc}
\usepackage{natbib}

\usepackage{cite}
\usepackage{amsmath,amssymb,amsfonts}
\usepackage{algorithmic}
\usepackage{graphicx}
\usepackage{textcomp}
\usepackage{multirow}
\usepackage{subcaption}
\usepackage{xcolor}
\def\BibTeX{{\rm B\kern-.05em{\sc i\kern-.025em b}\kern-.08em
    T\kern-.1667em\lower.7ex\hbox{E}\kern-.125emX}}

\bibliographystyle{plainnat}
\setcitestyle{aysep={}}


\geometry{margin=1in}

% Enable line numbering
\linenumbers

\newcommand\blfootnote[1]{
    \begingroup
    \renewcommand\thefootnote{}\footnote{#1}
    \addtocounter{footnote}{-1}
    \endgroup
}

\begin{document}

% Title
\title{GrapeCPNet: A deep learning integration for grape completion and phenotyping in 3D}
\author{
    Wenli Zhang \textsuperscript{a,*},
    Chao Zheng \textsuperscript{a},
    Chenhuizi Wang \textsuperscript{a},
    Pieter M. Blok \textsuperscript{b}, \\
    Haozhou Wang \textsuperscript{b},
    Wei Guo \textsuperscript{b}
    \blfootnote{Corresponding author: zhangwenli@bjut.edu.cn}
    \blfootnote{Email addresses: zhengchao97201@163.com (Chao Zheng), Wangchenhuizi@emails.bjut.edu.cn (Chenhuizi Wang), pieter.blok@fieldphenomics.com (Pieter M. Blok), haozhou-wang@g.ecc.u-tokyo.ac.jp (Haozhou Wang), guowei@g.ecc.u-tokyo.ac.jp (Wei Guo)}
}
\date{}

\maketitle

% Footnotes for authors
\noindent\textsuperscript{a} Information Department, Beijing University of Technology, Beijing, China \\
% Email: zhangwenli@bjut.edu.cn \\
\textsuperscript{b} Graduate School of Agricultural and Life Sciences, The University of Tokyo, Tokyo, Japan\\
% \textsuperscript{*} \\

\begin{abstract}
The measurement of phenotypic parameters of fresh grapes, especially at the individual berry level, is critical for yield estimation and quality control. 
Currently, these measurements are done by humans, making it costly, labor-intensive, and often inaccurate. 
Advances in 3D reconstruction and point cloud analysis allow extraction of detailed for grapes, yet current methods struggle with instance segmentation on incomplete point clouds due to occlusion. 
This study presents a novel deep-learning-based phenotyping pipeline designed specifically for 3D grape point cloud data.                                                         
First, individual berries are segmented from the grape bunch using the SoftGroup deep learning network. 
Next, a self-supervised point cloud completion network, termed GrapeCPNet, addresses occlusions by completing missing areas.  
Finally, morphological analyses are applied to extract berry radius and volumes. 
Validation on a dataset of four fresh grape varieties yielded  $R^2$ values of 85.53\% for berry radius and 96.89\% for berry volume, respectively. 
These results demonstrate the potential of the proposed method for rapid, practical acquisition of grape 3D phenotypic information in grape cultivation.
\end{abstract}

\textbf{Keywords:} 3D point cloud analysis, instance segmentation, shape completion

\doublespacing

\section{Introduction}

% check which citation works for you
\citep{barbole_comparative_2023}
%citet{2}


\section{Experimental}

\section{Experimental Results}

\subsection{Quantitative Analysis}

\section{Conclusion}


\section*{CRediT authorship contribution statement}
\textbf{author}: Writing – original draft, Visualization, Validation, Software, Investigation, Formal analysis, Data curation, Conceptualization. 

\section*{Declaration of competing interest}
The authors declare that they have no known competing financial interests or personal relationships that could have appeared to influence the work reported in this paper.

\section*{Data availability}
Data will be made available on request.

\section*{Acknowledgment}
This work was supported by

\bibliography{references}


\end{document}
